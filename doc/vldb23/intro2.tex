\section{Introduction}\label{sec:introduction}

Incremental view maintenance (IVM) is an important and well-studied problem in
databases~\cite{gupta-idb95}.  The IVM problem can be stated as follows: given a database $DB$ and
a view $V$ described by a query $Q$ that is a function of the database, i.e. $V = Q(DB)$,
maintain the contents of $V$ in response to changes of the database,
ideally more efficiently than by simply reevaluating $Q(DB)$ from scratch.  The goal is
to provide an algorithm that can evaluate $Q$ over the changes $\Delta DB$ applied
to the database, since in general the size of the changes is small $|\Delta DB| \ll |DB|$.

This paper provides a new perspective by proposing a new definition
of IVM based on a streaming model of computation\footnote{Our model is inspired by Digital Signal
Processing DSP~\cite{rabiner-book75}, applied to databases, hence the name \dbsp.}.  Whereas previous
IVM solutions are based on defining a notion of a (partial) derivative of $Q$ with respect to its inputs,
our definition only requires computing \emph{derivatives of streams} as functions of time.
Derivatives of streams are always well-defined if the data computed on has a notion of difference
that satisfies some simple mathematical properties --- specifically, that it forms a commutative
group.  (Fortunately, it has long been known that relational databases can be modeled
in such a way, e.g.~\cite{green-pods07, koch-pods10}.)

\dbsp has several attractive properties:

\begin{enumerate}[nosep, leftmargin=0pt, itemindent=0.5cm, label=\textbf{(\arabic{*})}]
\item it is \textbf{expressive}.  (a) It can be used to define
precisely multiple concepts: traditional queries, streaming computations, and incremental
computations.  (b) We have been able to express in \dbsp the full
relational algebra, computations over sets and bags,
nested relations, aggregation, flatmap, monotonic and nonmonotonic
recursion, stratified negation, while-relational programs, window queries,
streaming queries, streaming aggregation, and incremental versions of all
of the above.  In fact, we have built a \dbsp implementation of the
complete SQL language (\refsec{sec:implementation}).
\item it is \textbf{simple}.
\dbsp is built entirely on elementary concepts such as functions and algebraic groups.
\item mathematically \textbf{precise}.  All the results in this paper have been
formalized and checked using the Lean
proof assistant~\cite{moura-cade15}.
\item it is \textbf{modular}, in the following sense:
(a) the incremental version of a complex query can be reduced
recursively to incrementalizing its component subqueries.
This gives a simple, syntactic,
heuristic-free algorithm (Algorithm~\ref{algorithm-inc})
that converts an arbitrary \dbsp query plan to its incremental form.
(b) Extending \dbsp to support new primitive operators is easy,
and they immediately benefit from the rest of the theory of
incrementalization.
An important consequence of modularity is that the theory
can be efficiently implemented, as we
briefly discuss in \refsec{sec:implementation}.
\end{enumerate}

The core concept of \dbsp is the \emph{stream}, which is used to model changes
over time. We use $\stream{A}$ to denote the type of infinite streams with values of
type $A$. If $s \in \stream{A}$ is a stream,
then $s[t] \in A, t \in \mathbb{N}$ is the $t$-th element of $s$, also referred to as the \emph{value of the stream at time $t$}.
A streaming operator is a function that
consumes one or more streams and produces another stream.  We show
streaming computations with diagrams, also called ``circuits'',
where boxes are computations and streams are arrows.  The following diagram
shows a stream operator $T: \stream{A} \times \stream{B} \to \stream{C}$,
consuming two input streams $s_0$ and $s_1$
and producing one output stream $s$:

\begin{center}
\begin{tikzpicture}[auto,>=latex,minimum width=.5cm]
  \node[] (input0) {$s_0$};
  \node[below of=input0,node distance=.3cm] (dummy) {};
  \node[below of=dummy,node distance=.3cm] (input1) {$s_1$};
  \node[block, right of=dummy] (T) {$T$};
  \node[right of=T] (output) {$s$};
  \draw[->] (input0) -- (T);
  \draw[->] (input1) -- (T);
  \draw[->] (T) -- (output);
\end{tikzpicture}
\vspace{-.2cm}
\end{center}

We generally think of streams as sequences of ``small'' values,
such as insertions or deletions in a database.
However, we also treat the whole database as a \emph{stream of database
snapshots}.  We model a database as a
stream $DB \in \stream{SCH}$, where $SCH$ is the database schema.
Time is not wall-clock time, but counts
the transactions applied to the database.
Since transactions are linearizable, they have a total order.
$DB[t]$ is the snapshot of the
database contents after $t$ transactions have been applied.

Database transactions also form a stream $\Delta DB$, this time a stream of \emph{changes},
or \emph{deltas} that are applied to the database.  The values of
this stream are defined by $(\Delta DB)[t] = DB[t] - DB[t-1]$, where ``$-$'' stands
for the difference between two databases, a notion that we will soon make more precise.
The $\Delta DB$ stream is produced from the $DB$ stream by
the \emph{stream differentiation} operator $\D$;
this operator produces as its output the stream of changes from its input stream;
we have thus $\D(DB) = \Delta DB$.

Conversely, the database snapshot at time $t$ is the cumulative result of applying all
transactions up to $t$: $DB[t] = \sum_{i \leq t} \Delta DB[i]$.
The operation of adding up all changes is the inverse of differentiation,
and is another basic stream operator, \emph{stream integration}: $\I$.
The following diagram expresses the relationship between the streams $\Delta DB$ and $DB$:

\begin{center}
\begin{tikzpicture}[auto,>=latex,minimum width=.5cm]
  \node[] (input) {$\Delta DB$};
  \node[block, right of=input] (I) {$\I$};
  \node[right of=I] (output) {$DB$};
  \node[block, right of=output] (D) {$\D$};
  \node[right of=D] (end) {$\Delta DB$};
  \draw[->] (input) -- (I);
  \draw[->] (I) -- (output);
  \draw[->] (output) -- (D);
  \draw[->] (D) -- (end);
\end{tikzpicture}
\end{center}

Suppose we have a query $Q : SCH \to SCH$ defining a view $V$.  What is
a view in a streaming model?  It is also a stream!  For each snapshot
of the database stream we have a snapshot of the view: $V[t] = Q(DB[t])$.
In general, given an arbitrary function $f: A \to B$, we define
a streaming ``version'' of $f$, denoted by $\lift{f}$
(read as ``$f$ lifted''), which applies
$f$ to every element of the input stream independently.
We can thus write $V = (\lift{Q})(DB)$.

Armed with these basic definitions, we can now precisely define IVM.
What does it mean to maintain a view incrementally?  We claim that an
efficient maintenance algorithm needs to compute the \emph{changes} to
the view given the changes to the database. Given a query $Q$, a key
contribution of this paper is the definition of its \emph{incremental
  version} $\inc{Q}$ that uses stream integration and differentiation:
$\inc{Q} \defn D \circ \lift Q \circ I$. The incremental version of
the query maintains the changes to the view $\Delta V \defn \D(V) =
\D(\lift{Q}(\I(\Delta DB)))$, depicted graphically as:

\begin{center}
\begin{tikzpicture}[auto,>=latex,minimum width=.5cm]
  \node[] (input) {$\Delta DB$};
  \node[block, right of=input] (I) {$\I$};
  \node[block, right of=I, node distance=1.3cm] (Q) {$\lift{Q}$};
  \node[block, right of=Q, node distance=1.3cm] (D) {$\D$};
  \node[right of=D] (output) {$\Delta V$};
  \draw[->] (input) -- (I);
  \draw[->] (I) -- node (db) {$DB$} (Q);
  \draw[->] (Q) -- node (B) {$V$} (D);
  \draw[->] (D) -- (output);
\end{tikzpicture}
\end{center}

The incremental version of a query is a \emph{streaming operator}
which computes directly on changes and produces changes.  The
incremental version of a query is thus always well-defined.  The above
definition shows one way to compute a query incrementally, but
applying it naively produces an inefficient execution plan, since it
will reconstruct the database at each step.  In
\refsec{sec:incremental} we show how algebraic properties of the
$\inc{\cdot}$ transformation can be used to optimize the
implementation of $\inc{Q}$. The first key property is that the
composition of queries can be incrementalized by composing the
incremental versions of its constituents, that is $\inc{(Q_1 \circ
  Q_2)} = \inc{Q_1} \circ \inc{Q_2}$.  The second key property is that
essentially all primitive database operations have efficient
incremental versions\footnote{They are faster than non-incremental
versions by a factor of $O(|DB|/|\Delta DB|)$.}.

Armed with this general theory of incremental computation, in
\secref{sec:relational} we show how to model relational queries in
\dbsp.  This immediately gives us a general algorithm to compute the
incremental version of any relational query.  These results were
previously known, but they are cleanly modeled by \dbsp.
\secref{sec:datalog} shows how stratified-monotonic recursive Datalog
programs can be implemented in \dbsp, and \secref{sec:nested} gives
\emph{incremental streaming computations for recursive programs}. For
example, given an implementation of transitive closure in the natural
recursive way, our algorithm produces a program that efficiently
maintains the transitive closure of a graph as the graph is changed by
adding and deleting edges.

Due to space reasons, this paper omits proofs.  A companion extensive
technical report\anonymize{~\cite{tr}} contains the proofs.  We have
formalized the entire \dbsp theory in the Lean proof
assistant~\anonymize{\cite{dbsp-theory}}; our formalization includes
machine-checked proofs of correctness for all the theorems stated in
this paper.

This paper makes the following contributions:
\begin{enumerate}[nosep, leftmargin=0pt, itemindent=0.5cm, label=\textbf{(\arabic{*})}]
  \item \dbsp, a \textbf{simple} but \textbf{expressive} language for streaming
  computation. \dbsp gives an elegant formal foundation unifying the manipulation of
  streaming and incremental computations.
  \item An algorithm for incrementalizing any streaming computation expressed in
  \dbsp.
  \item An illustration of how \dbsp can model various query classes, such as relational algebra,
  nested relations, aggregations, flatmap, and stratified-monotonic Datalog.
  \item The first general and machine-checked theory of IVM.
  \item A high-performance open-source implementation of DBSP as a
  general-purpose streaming query engine in Rust.
\end{enumerate}
